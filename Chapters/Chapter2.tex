% Chapter 2

\chapter{Objetivos} % Main chapter title

\label{Chapter2} % Change X to a consecutive number; for referencing this chapter elsewhere, use \ref{ChapterX}

%----------------------------------------------------------------------------------------

\section{Objetivo General}

El objetivo general de este proyecto es la detección de enfermedades y plagas en el cultivo del datil medjool, mediante el uso de técnicas de visión por computadora y aprendizaje profundo.\\

%----------------------------------------------------------------------------------------

\section{Objetivos Específicos}

\begin{itemize}
    \item Implementar un sistema de detección de enfermedades y plagas en el cultivo del datil medjool.
    \item Evaluar el rendimiento del sistema de detección mediante métricas de precisión, recall y F1-score.
    \item Comparar el rendimiento del sistema de detección con otros métodos existentes en la literatura.
    \item Proponer mejoras al sistema de detección basado en los resultados obtenidos.
    \item Desarrollar una interfaz gráfica para la visualización de los resultados del sistema de detección.
    \item Realizar pruebas de campo para validar el sistema de detección en condiciones reales de cultivo.
\end{itemize}

%----------------------------------------------------------------------------------------

\section{Hipótesis}

Es posible usar redes neuronales convolucionales para el analisis del datil medjool, para la detección de humerar interna en el fruto así como enfermedades y plagas en el cultivo.\\


