% Chapter 1

\chapter{Descripción del proyecto}

\label{Chapter1}

%----------------------------------------------------------------------------------------

% Define some commands to keep the formatting separated from the content
\newcommand{\keyword}[1]{\textbf{#1}}
\newcommand{\tabhead}[1]{\textbf{#1}}
\newcommand{\code}[1]{\texttt{#1}}
\newcommand{\file}[1]{\texttt{\bfseries#1}}
\newcommand{\option}[1]{\texttt{\itshape#1}}

%----------------------------------------------------------------------------------------

\section{Introducción}

El cultivo del dátil Medjool ha cobrado una importancia cada vez mayor en el Valle de Mexicali y en otras regiones del noroeste de México. Su sabor, tamaño y valor comercial lo han convertido en una fruta apreciada tanto a nivel nacional como en los mercados internacionales \parencite{salomon-torres_produccion_2017}. Sin embargo, uno de los retos más evidentes al momento de su comercialización es asegurar la calidad del fruto de manera eficiente, especialmente cuando hablamos de aspectos internos como la humedad o posibles daños que no siempre se pueden ver a simple vista.\\

En la mayoría de las empacadoras, el proceso de selección y clasificación todavía depende en gran parte de la inspección manual, lo cual puede ser subjetivo y no siempre confiable \parencite{perez-perez_evaluation_2021}. Esto puede traducirse en pérdidas económicas, rechazo en aduanas o una mala experiencia para el consumidor final. Si bien existen tecnologías capaces de detectar con precisión la calidad interna de los frutos, como lo es la espectrocopía la cual ha demostrado ser efectiva en frutas como manzana, mango, piña o dátil \parencite{chen_prediction_2024, wang_improving_2025}. El alto costo mantiene este tipo de tecnologías aun fuera del alcance de muchos productores y pequeñas industrias.\\

Esta investigación propone desarrollar una alternativa automatizada basada en redes neuronales profundas, sensores ópticos e imágenes comunes (RGB), que permita analizar las características del fruto sin dañarlo. Se buscará entrenar modelos capaces de predecir información clave como el contenido de agua, la madurez y la posible presencia de defectos internos, integrando además datos colorimétricos bajo el modelo CIELab \parencite{habib_external_2022}.\\

Desde mi experiencia como ingeniero en sistemas embebidos y desarrollo con tecnologías abiertas, creo firmemente que este tipo de soluciones pueden ser aplicadas en campo de forma práctica y a bajo costo. Más allá de una simple clasificación visual, esta tesis busca sentar las bases para un sistema inteligente que pueda apoyar directamente al productor en su toma de decisiones, aumentar la eficiencia en la postcosecha y fortalecer el valor del dátil Medjool como producto agrícola de alta calidad.\\

%----------------------------------------------------------------------------------------

\section{Marco teórico}

\subsection{Producción del dátil Medjool}

El Medjool es una de las variedades de dátil con mayor aceptación en el mercado global, apreciado por su textura suave, tamaño grande y sabor dulce. En México, el Valle de Mexicali se ha consolidado como una de las principales zonas productoras, gracias a sus condiciones climáticas áridas y su experiencia agrícola. Sin embargo, a pesar del crecimiento en volumen y exportaciones, los métodos para evaluar la calidad del fruto siguen siendo manuales y dependientes de la experiencia del operador \parencite{salomon-torres_produccion_2017}.\\

Una de las limitantes más comunes en el proceso postcosecha es la dificultad para detectar problemas internos como fermentación, hongos o pérdida de humedad. Estos defectos no siempre se manifiestan en el exterior del fruto y pueden pasar desapercibidos hasta que el producto llega al consumidor. Además, la clasificación por madurez, que es esencial para determinar el destino comercial del dátil (fresco, industrial o exportación), también se realiza con base en la observación visual, lo que introduce variabilidad y reduce la eficiencia del proceso \parencite{perez-perez_evaluation_2021}.\\

\subsection{Inteligencia artificial aplicada a frutas}

El uso de algoritmos de visión computacional y aprendizaje automático ha crecido rápidamente en la agricultura, especialmente en tareas como clasificación de frutas, detección de defectos y predicción de calidad. Las redes neuronales convolucionales (CNNs), en particular, han demostrado una gran capacidad para identificar patrones visuales complejos como textura, forma, color y anomalías superficiales \parencite{albarrak_deep_2022, alsirhani_novel_2023}.\\

En el caso del dátil Medjool, modelos como los evaluados por \parencite{almomen_date_2023} han alcanzado precisiones superiores al 95\% utilizando imágenes RGB para clasificar el fruto según su etapa de madurez o estado superficial. Además, el aprendizaje por transferencia (transfer learning) ha facilitado el entrenamiento de estos modelos incluso con bases de datos reducidas, reutilizando redes como ResNet50 o MobileNet ya entrenadas en grandes conjuntos de imágenes.\\

La tendencia reciente se orienta hacia modelos híbridos, como los presentados por \parencite{said_smartripen_2025}, donde se integran CNNs con métodos de selección de características y regresores más tradicionales como XGBoost, logrando así sistemas más robustos y adaptables a múltiples variables.\\

\subsection{Técnicas ópticas no destructivas: NIR y colorimetría}

Además de las imágenes, otras fuentes de información han resultado útiles para evaluar la calidad interna de frutas sin destruirlas. Una de las más prometedoras es la espectroscopía en el infrarrojo cercano (NIR), que permite estimar parámetros como contenido de agua, azúcares o firmeza a partir de la respuesta óptica del fruto. Estudios como los de \parencite{chen_prediction_2024, wang_improving_2025} han validado modelos basados en NIR para predecir la humedad en manzanas y otros frutos, mostrando correlaciones significativas con datos obtenidos en laboratorio.\\

Sin embargo, estos sistemas suelen implicar equipos costosos. Por ello, se han comenzado a explorar alternativas de bajo costo disponibles en el mercado, que ofrece múltiples bandas en el rango visible e infrarrojo y puede integrarse fácilmente con microcontroladores o plataformas embebidas \parencite{passos_deep_2023}. Aunque su resolución es más limitada, su uso combinado con aprendizaje profundo puede ofrecer resultados aceptables en tareas específicas como la clasificación por contenido de humedad.\\

Por otro lado, el modelo de color CIELab ha sido utilizado con éxito para caracterizar frutas en distintas etapas de madurez. Este modelo cuantifica atributos visuales como luminosidad (L*), tonalidad roja-verde (a*) y azul-amarillo (b*), permitiendo correlaciones con la frescura o el deterioro del fruto \parencite{habib_external_2022}. Integrar esta información con imágenes RGB y espectros NIR permite desarrollar modelos multimodales que enriquecen la capacidad predictiva y reducen la dependencia de un solo tipo de dato.\\

\subsection{Aplicación local y perspectiva computacional}

Desde una perspectiva tecnológica, la integración de sensores accesibles, plataformas de procesamiento embebido y bibliotecas de código abierto (como TensorFlow o PyTorch), permite llevar estas soluciones más allá del laboratorio. Mi experiencia en desarrollo con tecnologías abiertas y visión por computadora busca facilitar la construcción de un sistema práctico y reproducible, que pueda adaptarse a las condiciones reales del campo en Baja California.\\

Este enfoque propone un medio para la adopción tecnológica gradual en la cadena de producción del dátil, facilitando decisiones informadas, reduciendo mermas y elevando el estándar de calidad del producto final.\\

%----------------------------------------------------------------------------------------

\section{Justificación}

La producción de dátil Medjool en el Valle de Mexicali se ha consolidado como una actividad agrícola de alta rentabilidad y proyección internacional. Sin embargo, conforme aumentan las exigencias del mercado, también crece la necesidad de garantizar estándares de calidad más rigurosos. Hoy en día, una parte considerable de la evaluación postcosecha como lo es la clasificación por madurez, la detección de defectos o la selección de frutos para exportación se sigue haciendo de forma visual y manual, lo que introduce errores, retrabajo y pérdidas económicas \parencite{perez-perez_evaluation_2021}.\\

Existen tecnologías capaces de detectar con precisión características internas de los frutos, como el contenido de humedad o la presencia de hongos, pero su alto costo y complejidad las hacen poco accesibles para pequeños y medianos productores. Esto crea una brecha entre el conocimiento técnico disponible y su aplicación real en el campo.\\

Este proyecto propone una alternativa accesible basada en redes neuronales profundas y sensores ópticos, integrando tres fuentes de información complementarias: imágenes RGB, colorimetría en espacio CIELab y espectros de luz en el rango cercano al infrarrojo (NIR). Si bien este tipo de soluciones ha demostrado ser efectiva en otros cultivos \parencite{chen_prediction_2024, wang_improving_2025}, la investigación y el desarrollo en el contexto del Medjool mexicano aún es limitada.\\

Lo que diferencia este trabajo es su enfoque en herramientas abiertas, económicas y reproducibles. El uso de módulos ópticos NIR accesibles compatibles con plataformas abiertas, cámaras convencionales y plataformas embebidas como computadoras de modulo individual permite construir un sistema funcional a bajo costo, con posibilidades reales de implementación en empacadoras de la región. Además, el modelo que se busca desarrollar no solo facilitaría la clasificación por madurez, sino que también permitiría detectar defectos internos o anticipar pérdidas por deshidratación, mejorando así la eficiencia del proceso postcosecha.\\

Desde mi formación en ingeniería y experiencia en sistemas embebidos, este proyecto representa una oportunidad para aplicar conocimientos técnicos de forma tangible en un sector estratégico para la región. A través de este enfoque, se espera contribuir al fortalecimiento de la agricultura local, promover la adopción tecnológica y generar un impacto positivo tanto económico como social.\\
