% Chapter 1

\chapter{Descripción del proyecto}

\label{Chapter1}

%----------------------------------------------------------------------------------------

% Define some commands to keep the formatting separated from the content
\newcommand{\keyword}[1]{\textbf{#1}}
\newcommand{\tabhead}[1]{\textbf{#1}}
\newcommand{\code}[1]{\texttt{#1}}
\newcommand{\file}[1]{\texttt{\bfseries#1}}
\newcommand{\option}[1]{\texttt{\itshape#1}}

%----------------------------------------------------------------------------------------

\section{Introducción}

El cultivo del dátil Medjool ha cobrado una importancia cada vez mayor en el Valle de Mexicali y en otras regiones del noroeste de México. Su sabor, tamaño y valor comercial lo han convertido en una fruta apreciada tanto a nivel nacional como en los mercados internacionales \parencite{salomon-torres_produccion_2017}. Sin embargo, uno de los retos más evidentes al momento de su comercialización es asegurar la calidad del fruto de manera eficiente, especialmente cuando hablamos de aspectos internos como la humedad o posibles daños que no siempre se pueden ver a simple vista.\\

En la mayoría de las empacadoras, el proceso de selección y clasificación todavía depende en gran parte de la inspección manual, lo cual puede ser subjetivo y no siempre confiable \parencite{perez-perez_evaluation_2021}. Esto puede traducirse en pérdidas económicas, rechazo en aduanas o una mala experiencia para el consumidor final. Si bien existen tecnologías capaces de detectar con precisión la calidad interna de los frutos, como lo es la espectrocopía la cual ha demostrado ser efectiva en frutas como manzana, mango, piña o dátil \parencite{chen_prediction_2024, wang_improving_2025}. El alto costo mantiene este tipo de tecnologías aun fuera del alcance de muchos productores y pequeñas industrias.\\

Esta investigación propone desarrollar una alternativa automatizada basada en redes neuronales profundas, sensores ópticos e imágenes comunes (RGB), que permita analizar las características del fruto sin dañarlo. Se buscará entrenar modelos capaces de predecir información clave como el contenido de agua, la madurez y la posible presencia de defectos internos, integrando además datos colorimétricos bajo el modelo CIELab \parencite{habib_external_2022}.

%----------------------------------------------------------------------------------------

\section{Marco teórico}

\subsection{Producción del dátil Medjool}

El Medjool es una de las variedades de dátil con mayor aceptación en el mercado global, algunas de sus características que lo distinguen son su textura suave, tamaño grande y sabor dulce. En México, particularmente la zona noroeste del país se ha consolidado como una de las principales zonas productoras, gracias a sus condiciones climáticas áridas y su experiencia agrícola. Sin embargo, a pesar del crecimiento en volumen y exportaciones, los métodos para evaluar la calidad del fruto siguen siendo manuales y dependientes de la experiencia del operador \parencite{salomon-torres_produccion_2017}.\\

Una de las limitaciones más comunes en el proceso postcosecha es la dificultad para detectar problemas internos como fermentación, hongos o pérdida de humedad. Estos defectos no siempre se manifiestan en el exterior del fruto y pueden pasar desapercibidos hasta que el producto llega al consumidor. Además, la clasificación por madurez, que es esencial para determinar el destino comercial del dátil (fresco, industrial o exportación), es realizado con base en la apreciación de los operadores, lo que introduce variabilidad y reduce la eficiencia del proceso \parencite{perez-perez_evaluation_2021}.

\subsection{Datasets aplicados a agricultura}

Un dataset es un conjunto estructurado de datos que contiene observaciones representadas por variables, las cuales pueden incluir entradas (features) y etiquetas (labels) asociadas. En el contexto de ciencia de datos y aprendizaje automático, los datasets constituyen la base fundamental para entrenar, validar y probar modelos computacionales \parencite{kamilaris_deep_2018}.\\

En estudios agrícolas, un dataset puede incluir imágenes del cultivo, espectros ópticos, parámetros físico-químicos, y anotaciones hechas por expertos. Su calidad, tamaño, y grado de variabilidad son factores determinantes en la capacidad de generalización de los modelos construidos a partir de ellos. Además, un dataset bien documentado permite la replicabilidad de resultados y la validación cruzada con datos de otros entornos o especies.

\subsection{Inteligencia artificial aplicada a frutas}

El uso de algoritmos de inteligencia artificial en los campos de la visión computacional y el aprendizaje automático ha crecido rápidamente en la agricultura, especialmente en tareas como la clasificación de frutas, la detección de defectos y la predicción de calidad. Las redes neuronales convolucionales (CNNs) han demostrado una gran capacidad para identificar patrones visuales complejos como textura, forma, color y anomalías superficiales \parencite{albarrak_deep_2022, alsirhani_novel_2023}.\\

En el caso del dátil Medjool, modelos como los evaluados por \parencite{almomen_date_2023} han alcanzado precisiones superiores al 95\% utilizando imágenes RGB para clasificar el fruto según su etapa de madurez o estado superficial. Además, el aprendizaje por transferencia (transfer learning) ha facilitado el entrenamiento de estos modelos incluso con bases de datos reducidas, reutilizando redes como ResNet50 o MobileNet ya entrenadas en grandes conjuntos de imágenes.\\

La tendencia reciente se orienta hacia modelos híbridos, como los presentados por \parencite{said_smartripen_2025}, donde se integran CNNs con métodos de selección de características y regresores más tradicionales como XGBoost, logrando así sistemas más robustos y adaptables a múltiples variables.

\subsection{Modelos unimodales y multimodales}

En el contexto del aprendizaje profundo, un modelo unimodal es aquel que se entrena con un solo tipo de entrada, como imágenes RGB, espectros NIR (Near Infra Red) o datos colorimétricos. Este enfoque presenta una menor complejidad en el diseño y entrenamiento del modelo, y puede ser adecuado cuando una sola modalidad contiene información suficiente para resolver la tarea planteada. Sin embargo, suele ser más sensible al ruido y menos generalizable si la información es parcial o está incompleta \parencite{kamilaris_deep_2018}.\\

Por el contrario, los modelos multimodales integran múltiples tipos de datos, como imágenes, espectros y variables numéricas, para capturar información complementaria y fortalecer la capacidad predictiva. Esta fusión puede realizarse en distintos niveles: a nivel de entrada (early fusion), en capas intermedias (intermediate fusion) o en la salida del modelo (late fusion) \parencite{passos_deep_2023}. En estudios agrícolas, se ha demostrado que la combinación de datos visuales y espectrales mejora la precisión en tareas como la detección de defectos internos, la estimación del contenido de agua y la clasificación de madurez de los frutos. Por ejemplo, en la predicción simultánea de características como los sólidos solubles totales (TSSC) y el contenido de humedad en cítricos, la integración de múltiples fuentes espectrales ha incrementado significativamente la exactitud de los modelos.\\

Este enfoque también mejora la robustez frente a condiciones variables de iluminación o entorno, un aspecto crítico en aplicaciones agrícolas reales. En consecuencia, el presente estudio adopta un enfoque multimodal que integrará imágenes RGB calibradas, espectros NIR y parámetros CIELab, permitiendo evaluar y comparar tanto modelos unimodales como multimodales en el contexto del dátil Medjool postcosecha.

\subsection{Técnicas ópticas no destructivas: NIR y colorimetría}

Además de las imágenes, otras fuentes de información han resultado útiles para evaluar la calidad interna de frutas sin destruirlas. Una de las más prometedoras es la espectroscopía en el infrarrojo cercano (NIR), que permite estimar parámetros como contenido de agua, azúcares o firmeza a partir de la respuesta óptica del fruto. Estudios como los de \parencite{chen_prediction_2024, wang_improving_2025} han validado modelos basados en NIR para predecir la humedad en manzanas y otros frutos, mostrando correlaciones significativas con datos obtenidos en laboratorio.\\

Sin embargo, estos sistemas suelen implicar equipos costosos. Por ello, se han comenzado a explorar alternativas de bajo costo disponibles en el mercado, que ofrece múltiples bandas en el rango visible e infrarrojo y puede integrarse fácilmente con microcontroladores o plataformas embebidas \parencite{passos_deep_2023}. Aunque su resolución es más limitada, su uso combinado con aprendizaje profundo puede ofrecer resultados aceptables en tareas específicas como la clasificación por contenido de humedad.\\

El modelo de color CIELab, propuesto por la Comisión Internacional de Iluminación (CIE) en 1976, representa los colores en un espacio tridimensional compuesto por tres coordenadas: L* (luminosidad), a* (eje verde-rojo) y b* (eje azul-amarillo). Este modelo fue diseñado para ser perceptualmente uniforme, lo que significa que pequeñas diferencias numéricas entre colores se corresponden con diferencias visuales igualmente perceptibles por el ojo humano. A diferencia del modelo RGB, que depende del dispositivo de captura y las condiciones de iluminación, CIELab permite realizar comparaciones objetivas y estandarizadas, lo que lo hace altamente útil en el análisis de calidad de alimentos \parencite{cen_theory_2007}.\\

En frutos como el dátil Medjool, el color superficial está estrechamente relacionado con su estado de maduración, pérdida de humedad y desarrollo de defectos. Cambios en L* reflejan oscurecimiento o pérdida de brillo, mientras que desplazamientos en a* y b* indican procesos como la degradación de clorofila o la oxidación de compuestos fenólicos \parencite{knott_facilitated_2023}. En este estudio, los parámetros CIELab serán obtenidos a partir de imágenes RGB calibradas y se integrarán con datos espectrales NIR en el desarrollo de modelos de aprendizaje profundo.\\

Esta combinación de datos visuales y espectrales constituye una estrategia multimodal robusta para predecir variables internas como el contenido de humedad o la presencia de defectos, incluso cuando no son visibles a simple vista. Estudios previos han demostrado que la inclusión del componente colorimétrico CIELab puede mejorar el rendimiento de modelos de clasificación automatizada, especialmente cuando se complementa con visión computacional e imágenes hiperespectrales \parencite{habib_external_2022}.

\subsection{Interpretabilidad de modelos con inteligencia}

La inteligencia artificial explicable (XAI, por sus siglas en inglés) hace referencia a un conjunto de métodos diseñados para interpretar y justificar las predicciones realizadas por modelos de aprendizaje automático, especialmente aquellos de tipo "caja negra" como las redes neuronales profundas. En lugar de limitarse a producir una predicción, los métodos XAI permiten comprender qué variables o patrones dentro de los datos de entrada influyeron más en la decisión del modelo \parencite{lundberg_unified_2017}.\\

Las técnicas XAI pueden clasificarse en dos grandes grupos: basadas en atribución, como SHAP (SHapley Additive exPlanations), que asignan un peso a cada característica de entrada en función de su impacto en la predicción; y basadas en visualización, como Grad-CAM (Gradient-weighted Class Activation Mapping), que permiten identificar las regiones más relevantes en imágenes al analizar activaciones dentro de redes convolucionales.\\

La inclusión de XAI en contextos agrícolas no solo aporta claridad sobre el funcionamiento interno del modelo, sino que también incrementa la confianza del usuario en las decisiones automatizadas, permite validar resultados con el conocimiento experto, y favorece la adopción de tecnologías basadas en aprendizaje automático en entornos reales.

\subsection{Aplicación local y perspectiva computacional}

Desde una perspectiva tecnológica, la integración de sensores accesibles, plataformas de procesamiento embebido y bibliotecas de código abierto (como TensorFlow o PyTorch), permite llevar estas soluciones más allá del laboratorio. Mi experiencia en desarrollo con tecnologías abiertas y visión por computadora busca facilitar la construcción de un sistema práctico y reproducible, que pueda adaptarse a las condiciones reales del campo en Baja California.\\

Este enfoque propone un medio para la adopción tecnológica gradual en la cadena de producción del dátil, facilitando decisiones informadas, reduciendo mermas y elevando el estándar de calidad del producto final.

%----------------------------------------------------------------------------------------

\section{Justificación}

La producción de dátil Medjool en el Valle de Mexicali se ha consolidado como una actividad agrícola de alta rentabilidad y proyección internacional. Sin embargo, conforme aumentan las exigencias del mercado, también crece la necesidad de garantizar estándares de calidad más rigurosos. Hoy en día, una parte considerable de la evaluación postcosecha como lo es la clasificación por madurez, la detección de defectos o la selección de frutos para exportación se sigue haciendo de forma visual y manual, lo que introduce errores, retrabajo y pérdidas económicas \parencite{perez-perez_evaluation_2021}.\\

Existen tecnologías capaces de detectar con precisión características internas de los frutos, como el contenido de humedad o la presencia de hongos, pero su alto costo y complejidad las hacen poco accesibles para pequeños y medianos productores. Esto crea una brecha entre el conocimiento técnico disponible y su aplicación real en el campo.\\

Este proyecto propone una alternativa accesible basada en redes neuronales profundas y sensores ópticos, integrando tres fuentes de información complementarias: imágenes RGB, colorimetría en espacio CIELab y espectros de luz NIR. Si bien este tipo de soluciones ha demostrado ser efectiva en otros cultivos \parencite{chen_prediction_2024, wang_improving_2025}, la investigación y el desarrollo en el contexto del Medjool mexicano aún es limitada.\\

Lo que diferencia este trabajo es su enfoque en herramientas abiertas, económicas y reproducibles. El uso de módulos ópticos NIR accesibles compatibles con plataformas abiertas, cámaras convencionales y plataformas embebidas como computadoras de módulo individual permite construir un sistema funcional a bajo costo, con posibilidades reales de implementación en empacadoras de la región. Además, el modelo que se busca desarrollar no solo facilitaría la clasificación por madurez, sino que también permitiría detectar defectos internos o anticipar pérdidas por deshidratación, mejorando así la eficiencia del proceso postcosecha.