% Chapter 4

\chapter{Cronograma y presupuesto}

\label{Chapter4}

%----------------------------------------------------------------------------------------

\section{Cronograma de actividades}

El siguiente cronograma organiza las actividades del proyecto a lo largo de ocho semestres (cuatro años), considerando el desarrollo del sistema, adquisición de datos, entrenamiento de modelos, validación y publicación de resultados.\\

\begin{landscape}
\begin{table}[h]
\centering
\scriptsize
\begin{tabular}{|p{10cm}|c|c|c|c|c|c|c|c|}
\hline
\textbf{Actividad} & \textbf{S1} & \textbf{S2} & \textbf{S3} & \textbf{S4} & \textbf{S5} & \textbf{S6} & \textbf{S7} & \textbf{S8} \\
\hline
Revisión bibliográfica y entrevistas con productores \newline [OE1] Marco conceptual & $\blacksquare$ & $\blacksquare$ &  &  &  &  &  &  \\
\hline
Identificación de variables y criterios de evaluación \newline [OE1] Objetivos del protocolo definidos y justificados & $\blacksquare$ & $\blacksquare$ &  &  &  &  &  &  \\
\hline
Redacción de artículo de revisión \newline [OE1] Artículo de revisión & $\blacksquare$ & $\blacksquare$ &  &  &  &  &  &  \\
\hline
Recolección de muestras y etiquetado \newline [OE2] Dataset preliminar &  & $\blacksquare$ & $\blacksquare$ & $\blacksquare$ &  & $\blacksquare$ &  &  \\
\hline
Aplicación de mediciones físico-químicas \newline [OE2] Etiquetas de referencia (ground truth) &  & $\blacksquare$ & $\blacksquare$ & $\blacksquare$ &  & $\blacksquare$ &  &  \\
\hline
Redacción de artículo metodológico sobre dataset y adquisición \newline [OE2, OE3] Artículo de dataset y metodología &  &  & $\blacksquare$ & $\blacksquare$ &  &  &  &  \\
\hline
Preprocesamiento e integración de los datos en un dataset estructurado \newline [OE2, OE3] Dataset multimodal (publicable) &  &  & $\blacksquare$ & $\blacksquare$ &  & $\blacksquare$ &  &  \\
\hline
Análisis exploratorio de correlación entre variables ópticas y fisicoquímicas \newline [OE2] Análisis para artículo metodológico &  &  & $\blacksquare$ & $\blacksquare$ &  &  &  &  \\
\hline
Diseño, entrenamiento y comparación de modelos unimodales y multimodales \newline [OE3, OE4] Resultados comparativos entre modelos &  &  &  & $\blacksquare$ & $\blacksquare$ &  &  &  \\
\hline
Evaluación de desempeño e interpretabilidad (XAI) de los modelos \newline [OE4, OE5] Explicaciones y visualizaciones de interpretabilidad &  &  &  &  & $\blacksquare$ & $\blacksquare$ &  &  \\
\hline
Redacción de artículo de resultados de modelos y evaluación \newline [OE3, OE4, OE5] Artículo de resultados del modelo &  &  &  &  & $\blacksquare$ & $\blacksquare$ &  &  \\
\hline
Desarrollo del prototipo funcional y validación en entorno experimental \newline  [OE6] Prototipo de validación / informe técnico &  &  &  &  &  & $\blacksquare$ & $\blacksquare$ &  \\
\hline
Preparación de resultados para difusión y presentación de tesis \newline  [OE6] Tesis completa / presentaciones / informes &  &  &  &  &  &  & $\blacksquare$ & $\blacksquare$ \\
\hline
\end{tabular}
\caption{Cronograma de actividades por semestre.}
\label{tab:cronograma}
\end{table}
\end{landscape}

%----------------------------------------------------------------------------------------

\section{Presupuesto estimado}

El siguiente presupuesto contempla los recursos necesarios para desarrollar el enfoque multimodal propuesto, incluyendo adquisición de datos, pruebas de laboratorio, análisis computacional, difusión científica y trabajo de campo. Los precios fueron consultados en línea en abril y mayo de 2025, y se estimaron considerando el costo promedio en México.\\

\begin{table}[h]
\centering
\begin{tabular}{|p{5cm}|r|p{6cm}|}
\hline
\textbf{Concepto} & \textbf{Costo estimado (MXN)} & \textbf{Justificación} \\
\hline
Sensor óptico multiespectral (NIR) & 17000 & Medición espectral NIR no destructiva para estimar humedad y defectos internos. \\
\hline
Cámara RGB industrial o científica & 6000 & Captura de imágenes calibradas para análisis de color y textura (CIELab y RGB). \\
\hline
Iluminación LED calibrada & 2000 & Controlar condiciones de iluminación para estandarizar adquisiciones ópticas. \\
\hline
Plataforma de procesamiento (Google Coral Dev Board / Jetson Nano / LattePanda 3 Delta) & 4000 & Procesamiento de datos ópticos en campo o laboratorio con hardware embebido compatible con visión computacional. \\
\hline
Texturómetro o penetrómetro digital & 5500 & Medición precisa de firmeza para establecer etiquetas de referencia del fruto. \\
\hline
Materiales para corte y conservación de muestras & 2500 & Preparar y conservar muestras para evaluación interna y laboratorio. \\
\hline
Acceso a horno de secado / equipo de laboratorio & 7000 & Secado controlado para determinación de humedad como estándar de referencia. \\
\hline
Transporte y logística para recolección de muestras & 4000 & Recolección estacional de muestras desde huertos en Mexicali. \\
\hline
Traslados Ensenada–Mexicali (4 viajes de campo) & 4800 & Visitas de campo durante temporadas de cosecha en el Valle de Mexicali. \\
\hline
Publicación en revistas científicas (3 artículos) & 30000 & Cubrir cuotas de publicación en revistas de acceso abierto indexadas. \\
\hline
Material de difusión técnica y presentaciones & 3000 & Carteles, impresiones y recursos para presentación en congresos. \\
\hline
Gastos imprevistos y soporte técnico & 3000 & Mantenimientos menores, insumos adicionales o sustitución de componentes. \\
\hline
\textbf{Total} & \textbf{88800} & \\
\hline
\end{tabular}
\caption{Presupuesto detallado.}
\label{tab:presupuesto}
\end{table}

Los precios fueron consultados en los sitios web de proveedores como Digi-Key, SparkFun, Mouser, Amazon México, AliExpress, Equipar, Cytron y MercadoLibre. Se consideró el rango promedio de costos para versiones funcionales compatibles con sensores NIR, cámaras calibradas, y plataformas embebidas. El presupuesto de viajes se calculó tomando en cuenta el costo promedio por viaje redondo Ensenada–Mexicali considerando transporte, alimentos y posibles estancias. Las cuotas de publicación fueron estimadas a partir de los sitios oficiales de editoriales como Elsevier, MDPI y Springer para artículos open access en revistas indexadas.\\

\textit{Nota: Se contempla complementar el financiamiento con convocatorias estatales o federales, colaboración institucional y aportes en especie por parte de productores del Valle de Mexicali.}