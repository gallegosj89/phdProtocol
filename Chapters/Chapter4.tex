% Chapter 4

\chapter{Cronograma y presupuesto}

\label{Chapter4}

%----------------------------------------------------------------------------------------

\section{Cronograma de actividades}

El siguiente cronograma organiza las actividades del proyecto a lo largo de ocho semestres (cuatro años), considerando el desarrollo del sistema, adquisición de datos, entrenamiento de modelos, validación y publicación de resultados.

\begin{center}
\begin{tabular}{|p{10cm}|c|c|c|c|c|c|c|c|}
\hline
\textbf{Actividad} & \textbf{S1} & \textbf{S2} & \textbf{S3} & \textbf{S4} & \textbf{S5} & \textbf{S6} & \textbf{S7} & \textbf{S8} \\
\hline
Revisión bibliográfica y estado del arte & $ \blacksquare$ &  $ \blacksquare$ & & & & & & \\
Diseño experimental y planificación & $\blacksquare$ & $\blacksquare$ & $\blacksquare$ & & & & & \\
Recolección y etiquetado de muestras & $\blacksquare$ & $\blacksquare$ & $\blacksquare$ & $\blacksquare$ & & & & \\
Adquisición de datos (RGB, NIR, CIELab) & $\blacksquare$ & $\blacksquare$ & $\blacksquare$ & $\blacksquare$ & & & & \\
Procesamiento y organización del dataset & & $\blacksquare$ & $\blacksquare$ & $\blacksquare$ & $\blacksquare$ & & & \\
Desarrollo de modelos individuales & & & $\blacksquare$ & $\blacksquare$ & $\blacksquare$ & & & \\
Desarrollo de modelos multimodales & & & & $\blacksquare$ & $\blacksquare$ & $\blacksquare$ & & \\
Validación e interpretación de resultados (XAI) & & & & & $\blacksquare$ & $\blacksquare$ & $\blacksquare$ & \\
Redacción del artículo de revisión & $\blacksquare$ & $\blacksquare$ & & & & & & \\
Publicación del dataset & & & $\blacksquare$ & $\blacksquare$ & $\blacksquare$ & & & \\
Redacción de artículos científicos & & & & $\blacksquare$ & $\blacksquare$ & $\blacksquare$ & & \\
Redacción y presentación de tesis & & & & & & & $\blacksquare$ & $\blacksquare$ \\
Difusión y colaboración con productores & & & & $\blacksquare$ & $\blacksquare$ & $\blacksquare$ & $\blacksquare$ & $\blacksquare$ \\
\hline
\end{tabular}
\end{center}


%----------------------------------------------------------------------------------------

\section{Presupuesto estimado}

El siguiente presupuesto contempla los recursos mínimos necesarios para llevar a cabo el proyecto utilizando herramientas de bajo costo, plataformas de código abierto y colaboración con instituciones y productores locales.

\begin{center}
\begin{tabular}{|p{10cm}|r|}
\hline
\textbf{Concepto} & \textbf{Monto (MXN)} \\
\hline
Sensores ópticos e iluminación controlada & \$15,000 \\
Plataforma embebida (Raspberry Pi / Jetson Nano) & \$4,000 \\
Equipamiento de captura y estructura de muestreo & \$6,000 \\
Análisis físico-químicos (laboratorio) & \$10,000 \\
Insumos y logística de campo & \$6,000 \\
Publicación de artículos científicos (Open Access) & \$30,000 \\
Material de divulgación y difusión técnica & \$3,000 \\
Gastos imprevistos y mantenimiento & \$4,000 \\
\hline
\textbf{Total estimado} & \textbf{\$78,000} \\
\hline
\end{tabular}
\end{center}

\textit{Nota: Se contempla complementar el financiamiento con convocatorias estatales o federales, colaboración institucional y aportes en especie por parte de productores del Valle de Mexicali.}
